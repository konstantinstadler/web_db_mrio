\section{Conclusion and outlook}

Here we described an integrated web-interface, the Environmental Footprints Explorer, developed to ease the access to EE MRIO results. The platform provides functionality for the in depth exploration of a single database as well as for the comparison across several EE MRIO databases.

Although several EE MRIO have been developed and published in the past couple of years, the use of their results in informing the policy development process has been rather limited. The reason for that being are that on one hand that some expertise are needed to calculate indicators based on EE MRIO accounting, on the other hand the often overwhelming amount of data provided by current EE MRIO databases.

Some EE MRIO projects have tried to overcome that problem in the past by providing various kind of open access result reporting. For example, the OPEN:EU projects provides the policy tool EUREAPA to allow easy access to environmental and economic data for an EE MRIO model based on GTAP \cite{Roelich_2014}. The EE MRIO Eora \cite{Lenzen_2013} includes a web-interface to access the main results (http://worldmrio.com) and the EXIOBASE project provide a booklet summarizing the main results for a policy audience \cite{tukker_global_2014}. Similar initiatives exist also on the national level, for example the Open IO-Canada allows to retrieve environmental information based on the Canadian economic input-output tables (http://ciraigdev.polymtl.ca). However, all these initiatives fall short in enabling a comparison across databases.  

The overall goal of the Environmental Footprints Explorer is to deliver the newest environmental accounting results to policy makers and the public. Previously, these data was filtered by the analyst of the databases and often restricted to the top level results considered novel enough for a scientific publication. Detailed data or the newest updates of previously published data may still be hidden in the databases, although they could be critical for informing targeting policy development.

Recently, the European Science Policy (http://ec.europa.eu/programmes/horizon2020/en/h2020-section/open-science-open-access) as well as various scientific journals \cite{Hanson_2011, Stodden_2012, Boulton_2012} committed to a open science and data scheme. We feel, that, in an ideal situation, open science should not only provide open access to the raw data used for any analysis, but also provide a way to utilize the data for non-experts. The Environmental Footprints Explorer represents a first step in that direction.
    
    