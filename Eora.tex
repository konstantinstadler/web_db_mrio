\subsubsection{Eora}


Eora is a time-series (1990-2011) of input-output tables with high country detail (187 countries and 1 RoW region), utilizing asymmetric levels of detail and mixed IO and Supply-Use table structure \cite{Lenzen_2012, Lenzen_2013}. There is further a smaller version of Eora at 26 sectors common for all countries. The total database has 15,909 sectors with the size of the system having just over 250 million variables. The database has 35 types of environmental indicators covering air pollution, energy use, greenhouse gas emissions, water use, Ecological Footprint, and Human Appropriation of Net Primary Productivity. Natively the Eora MRIO uses heterogeneous classifications, which allows higher detail in major economies, but at the cost of added complexity for inter-country comparative analyses. A follow-on project from Eora, the Industrial Ecology Virtual Laboratory, continues in this direction, adding increased spatial and sectoral detail but at the cost of increased structural complexity \cite{Lenzen_2014}.

  