Calculation of EE MRIO accounts (Multipliers, Footprints, impacts embodied in trade, etc) followed
standard IO methodology. 

All results are  calculated through input-output analysis (IOA) following classic Leontief demand style modeling and are applicable for analysing historic data. IOA is essentially an allocation of production based impacts (here denoted \textbf{F}) to the goods and services that flow to final demand (here denoted \textbf{Y}). The allocation starts from the basic production balance, where gross output \textbf{x} is the sum of total intermediate demand \textbf{T} and final demand. 
\begin{equation}
    x=T+y
\end{equation}
Normalising intermediate production (to produce one unit of output, we can calculate the required inputs, giving the coefficient matrix A)
\begin{equation}
    A=T{x^-1}
\end{equation}
Then we can combine the above two equations, using what has been known as the Leontief inverse, to estimate the total output for a given demand
\begin{equation}
x*={(I-A)^-1}y*
\end{equation}
In MRIO modelling, for a certain demand, in a certain country, the production required to satisfy the demand is calculated. The point of departure from national accounting in gross domestic product terms is that imports are endogenised in the flows of goods to demands, and exports are excluded from a country’s demand – in line with gross national expenditure calculations. A second point of departure from available input-output tables is that generalizing the input-output table for environmental inputs or emissions to the environment is required. Similar to the \textbf{A} matrix of intermediate coefficients, a matrix of environmental interventions per unit output \textbf{S} is used to calculate the overall environmental impact \textbf{F*} associated with a certain demand \textbf{y*} 
\begin{equation}
F*=SLy*
\end{equation}
