Although several EE MRIO have been developed and published in the past couple of years, the use of their results in informing the policy development process has been rather limited. 

The reason for that being are that on one hand that some expertise are needed to calculate indicators based on EE MRIO accounting, on the other hand the often overwhelming amount of data provided by current EE MRIO databases.

Some EE MRIO projects have tried to overcome that problem in the past by providing various kind of open access result reporting. For example, the OPEN:EU projects provides the policy tool EUREAPA to allow easy access to environmental and economic data for an EE MRIO model based on GTAP \cite{Roelich_2014}. The EE MRIO Eora \cite{Lenzen_2013} includes a web-interface to access the main results (http://worldmrio.com) and the EXIOBASE project provide a booklet summarizing the main results for a policy audience \cite{tukker_global_2014}. Similar initiatives exist also on the national level, for example the Open IO-Canada allows to retrieve environmental information based on the Canadian economic input-output tables (http://ciraigdev.polymtl.ca). However, all these initiatives fall short in enabling a comparison across databases.  

