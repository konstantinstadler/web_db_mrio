Principally, pressures arising from economic activity can either be tallied at
the actual place of production or at the point of the consumption.
Traditionally, the accounting at the actual place of production has been used
for policy decisions. For example, the Kyoto Protocol was based on the
production principle, which is now discussed to be one of the reasons for its
limited success \cite{23192129}. 

Consumption based accounting (CBA), on the other hand, just recently gained
momentum for policy making \cite{Harris_2013}. In CBA, all resource use and environmental
pressurs (for example emissions) occurring along the the supply chain of
products are added up and allocated to the final consumer of the product. This
allows for example to assess the environmental pressure caused by a domestic
demand abroad \cite{Weinzettel_2013, 20212122}.

Indicators based on consumption based accounting are also known as various
types of 'footprints' and are increasingly used for informing policy makers as
well as the public about ongoing environmental \cite{tukker_global_2014, \cite{steen-olsen_carbon_2012, Kanemoto_2014, \cite{united_nations_university_inclusive_2012}
and social problems \cite{Simas_2014}. 

CBA requires a comprehensive description of the global socio-economic
metabolism including among other the structure of domestic economies, their
ressource use and the amount of trade between countries and regions.
Environmentally Extended Multi Regional Input-Output databases (EEMRIO) provide
all these data in a consistent framework. 