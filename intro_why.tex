Evidence guided policy making aiming for lessening the negative environmental and social impacts of our society require a comprehensive accounting principle. Such a principle must not only take into account direct domestic emissions and resource usage but also make the connection across global production networks - ultimately we are often interested in what forms of consumption have driven the impacts caused in production processes.

Global Environmentally Extended Multi Regional Input-Output (EE MRIO) tables provide such an system by taking into account the interrelations between production and consumption. EE MRIO tables link supply chains from the source of an impact, across countries, across processing stages, and to the final consumer. As such, estimates can be made about the amount of greenhouse gas emissions occurring in China to produce steel used in European televisions, or the amount of land used in Brazil required to feed British consumers. An important aspect of EE MRIO tables is that they are macro level economy-wide systems that provide a holistic picture of global production and consumption.    

During the last years, several EE MRIO databases have been published \cite{Tukker_2013}. These databases differ in their environmental and regional focus and also due to the applied accounting principles and implementation details \cite{Stadler_2014, Owen_2014, Moran_2014}. 
