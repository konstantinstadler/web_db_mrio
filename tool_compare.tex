\subsection{Comparing accounts across databases in a common classification system}

The main problem in any EE MRIO comparison is the sector and accounting
mismatch between the databases. To solve that issue, we aggregated every
database into a Common Classification System \cite{Steen_Olsen_2014}. 

Leaving the model field in the selection unspecified allows to compare results across databases. 
In addition, we compiled one additional database based on the average results of all included MRIOs.
As such, the Environmental Footprints Explorer provides a consistent way to access environmental accounts of different EE MRIO databases and to compare these results across databases.

The implementation of the Environmental Footprints Explorer allows to easily extend the number of EE MRIO databases accessible. In the short term, EXIOBASE 2, WIOD, EORA and OPEN:EU  have been integrated. We plan to extend the list of available MRIO as further databases become publicly available. 

Some recent studies analyzed the differences between EE MRIO databases \cite{Stadler_2014, Owen_2014, Moran_2014}. These studies provided important background information on the reasons of the observed differences and some exemplified comparisons on the effect of the differences of indicators. In order to get the amount of difference for other indicators, any practitioner would need to redo the whole study, starting from parsing several EE MRIO databases, aggregating to a common system and re-analyze the indicator results. 
The proposed web-platform provides an alternative approach. The comparison of the EE MRIO databases is readily available through the user interface as is the access to the detailed results (Figure \ref{fig:table})..
